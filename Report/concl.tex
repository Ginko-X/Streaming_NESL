\chapter{Conclusion}
Based on the practical experimentation that has demonstrated good performance and time and space efficiency from previous work of SNESL \cite{Fphd}, this thesis has moved  the development of SNESL one step forward. 

The main contributions of this thesis are:
\begin{itemize}
	\item Extension of the dataflow model of streaming to account for recursion. \\
	The challenge of supporting recursion is that it can cause an infinite increase of the dataflow network in the execution model.
	Our solution is to extend the target language to make sure that the dataflow graph will be completed dynamically during execution, but not grow infinitely, if all recursions are guarded by conditionals (i.e., comprehensions in SNESL). 
	We have developed the implementation as three instrumented interpreters, the high-level one for \mysnesl, the eager one with sufficiently large space  and the streaming one with a limited buffer,  to compare both results and costs. 
	Although without a formal proof, we have demonstrated result and cost preservation in our experiments, and provided some representative examples.
	The space usage from our experiment results also shows an increase proportional to the depth of the recursion. 
	
\item A formalization of the source and target languages, and the correctness of the translation including work cost preservation.  \\
Formal semantics for the high-level NDP language can be found from previous related research, such as NESL and Proteus. However, none of them has given a formal semantics of the target language. 
This thesis has presented a formalization of the target language for a core subset of SNESL, and also given a detailed proof for the correctness of translation and work cost preservation, as well as some other important properties of the language.
The work we have done in this thesis should lay out the possibility for a formal validation of the translation correctness and cost preservation for full SNESL.
\end{itemize}

While investigating the solution to recursion, we have also touched some crucial, open problems in this streaming language,  such as streamability, scheduling, and deadlocks. 
We have explored a possibility of scheduling that can be as efficient and light as the one used in the previous work of SNESL, but also preserves the cost.

The future work on SNESL related to the scope of this thesis fall into the following two points.
\begin{itemize}
		\item Formalization of the streaming semantics of the target language. 
		The formalization work we have done in this thesis is only for its eager semantics, while the streaming semantics is also an area that has not been covered much yet.
		
	\item Schedulability.  As a streaming language aiming at both time and space efficiency,  SNESL should be equipped with a type system or some static/dynamic analysis that can prevent a measure of problematic programs. The user may expect that, at least for a streamable program, deadlock will not happen. This will require a good high-level characterization of streamability, and a formal demonstration that all streamable programs indeed execute without deadlocks.
	

\end{itemize}

